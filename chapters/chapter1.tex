%-*-coding: utf-8-*-

\chapter{Обзор предметной области}

\section{Вспомогательные понятия}

\subsection{Vector representation и Word embedding}
Vector representation (Векторное представление)~--- подход в машинном обучении, при котором некоторой сущности
сопоставляется вектор вещественных чисел.

Формально, $X$~--- множество объектов, тогда фунция $v(x):X -> R^n$
задает vector representation для объектов из множества $X$.

Причем похожим сущностям сопоставляются близкие
по некоторой метрике вектора, а различным ~--- удаленные друг от друга.

Word embedding ~--- это подход в Natural Language Processing (NLP), который
состоит в отображении слов некоторого словаря в $R^n$ с сохранением
семантических отношений между словами.

Word embedding обычно предобучают на достаточно большом корпусе текстовых
данных, а затем используют в задачах NLP.

%Визуализация Word Embedding

\subsection{Sentence modeling}

\section{Классификация}

Классификация ~-- один из разделов машинного обучения, посвященный решению
задачи классификации.

\subsection{Задача классификации}

Задача классификации~--- имеется множество объектов, каждый из которых принадлежит
к какому-то классу, количество классов чаще всего ограничено.
Существует обучающая выборка~--- множество объектов, метки
класса которых нам известны. Классовая принадлежность остальных объектов
неизвестна. Задача заключается в построении алгоритма, способного
классифицировать (присвоить метку класса) произвольный объект из исходного множества.

Формально, $X$~--- множество объектов, $Y$~--- множество классов,
существует отношение $y* : X \rightarrow Y$, заданное только для обучающей выборки.
Необходимо построить такой алгоритм $a: X \rightarrow Y$, способный для произвольного
$x \in X$ найти $y \in Y$.

\subsection{Sentiment classiciation}

\section{Численная оценка качества классификации}

\subsection{Accuracy}

\subsection{Cross entropy loss}

1.4 
