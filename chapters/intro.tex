\startprefacepage

Искусственные нейронные сети (ИНС) являются одним из мощных инструментов машинного
обучения.
Это математическая модель, которая работает подобно тому, как работает головной
мозг человека \cite{rosenblatt58a}.
Они были изобретены в 60-х годах предыдущего века [статья], и нашли свое активное
применение в наши дни. Конечно ИНС претерпевали изменения и модернизации
[статьи], было разработано множество архитектур[статьи], а также алгоритмов обучения[статьи].
Нейронные сети успешно решают задачи анализа и обработки изображений, текста и
звука.

%Актуальность
В последнее время ИНС используются для множества задач, связанных с Natural
Language Processing (NLP). Важной задачей NLP является задача 
обработки и извлечения семантического содержимого предложений (sentence modelling).
Она используется для создания искусственного интеллекта, чат-ботов,
переводов текстов (machine translation), извлечение фактов из текста (information
extraction), схожесть утверждений (semantic relatedness), 
а также различных классификаций текстов, например,
по стилю или по эмоциональному содержанию (sentiment analysis).
В частности, широкое распространение получили методы, которые сопоставляют предложению некоторый вещественный вектор.

%Новизна
На данный момент существуют подходы, которые явно используют дерево синтаксического разбора предложения (то есть зависимости между словами в приложении), а также подходы, которые не используют его. Большинство решений полагается на априорные знания о каждом слове, и на основе этого строится дальнейшая модель.
В рамках данной работы будут предложены алгоритмы, которые используют дерево синтаксического разбора, а также учитывают локальный контекст каждого слова предложения. На данный момент не существует решений, которые явно учитывают оба этих аспекта.

% Структура работы
В главе 1 будут введены вспомогательные понятия, сформулированы решаемые задачи.
Будут рассмотрены существующие решения на основе CNN, 
LSTM, а также такие решения как RNTN, Tree-LSTM, Paragraph Vector.

В главе 2 будут рассмотрены предложенные решения, их принцип работы и обоснование.

В главе 3 будут рассмотрены результаты достигнутые этими решениями на различных задачах, а также сравнения с существующими решениями.