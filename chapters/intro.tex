\startprefacepage
В нашу эпоху современных технологий, человек пытается максимально
роботизировать все процессы нашей жизни. Ученые работают над тем,
чтобы компьютер без проблем понимал, что от него хочет пользователь и с легкостью
решал поставленные задачи. 
Для того, чтобы компьютер мог понимать человеческий язык, было изобртено направление искусственного интеллекта~--- обработка естественного языка (от англ. Natural Language Processing). 
NLP изучает проблемы анализа и синтеза естественных языков\cite{wikinlp}.
Большую нишу NLP заняли искусственные нейронные сети.

Искусственные нейронные сети (ИНС) являются одним из мощных инструментов машинного
обучения.
Это математическая модель, которая работает подобно тому, как работает головной
мозг человека \cite{rosenblatt58a}.
Они были изобретены в 60-х годах предыдущего века, и нашли свое активное
применение в наши дни. Конечно ИНС претерпевали изменения и модернизации, было разработано множество архитектур[статьи], а также алгоритмов обучения\cite{Duchi2011, zeiler2012, rprop93}.

%Актуальность
Важной задачей NLP является задача  обработки и 
извлечения семантического содержимого предложения (sentence modelling).
Она используется для создания чат-ботов, переводов текстов (machine translation), 
извлечение фактов из текста (information extraction), схожесть утверждений (semantic relatedness), 
а также различных классификаций текстов, например, по стилю или по эмоциональному тону (sentiment analysis).
В частности, широкое распространение получили методы, 
которые сопоставляют предложению некоторый вещественный вектор, 
с помощью которого и происходит анализ предложения.

%Новизна
В данной работе будет предложена модель, вычисляющая векторное представление предложения, 
которая использует \emph{дерево синтаксического разбора} (syntactic parse tree) предложения, а также учитывают локальный контекст каждого слова предложения.
На данный момент не существует решений, которые явно учитывают оба этих аспекта.
Также будет предложен новый подход вычисления векторного представления предложения.

% Структура работы
В главе 1 будут введены вспомогательные понятия, сформулированы решаемые задачи.
Будут рассмотрены существующие решения на основе \emph{cверточных нейронных сетей} (от англ. Сonvolutional Neural Networks или CNN), \emph{рекурсивных нейронных сетей} (от англ. Recursive Neural Network или RNN), а также такие решения как Paragraph Vector и \emph{рекурсивная тензорная нейронная сеть} (от англ. Recursive Neural Tensor Network или RNTN).

В главе 2 будут рассмотрены предложенные улучшения, их принцип работы и обоснование.

В главе 3 будут рассмотрены результаты, достигнутые предложенными решениями на задачах классификации
эмоционального тона предложения и задача классификации вопросов, а также сравнение с существующими решениями.