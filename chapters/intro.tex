\startprefacepage

Искусственные нейронные сети (ИНС) являются одним из мощных инструментов машинного
обучения.
Это математическая модель, которая работает подобно тому, как работает головной
мозг человека[статья].
Они были изобретены в 60-х годах предыдущего века [статья], и нашли свое активное
применение в наши дни. Конечно ИНС претерпевали изменения и модернизации
[статьи], было разработано множество архитектур[статьи], а также алгоритмов обучения[статьи].
Нейронные сети успешно решают задачи анализа и обработки изображений, текста и
звука.

В последнее время ИНС используются для множества задач, связанных с Natural
Language Processing (NLP), таких как ... . Важными задачами NLP являются задачи
связанные обработкой и извлечением семантического контекста предложений и
текстов.
Они используются, например, для создания Искусственного интеллекта, а также чат-ботов,
которые в последнее время получили активное распространение.

Одной из таких задач является задача сопоставления предложению некоторого
вещественного вектора. Это основопологающая часть для следующих
задач: перевод текста (англ), извлечение фактов из текста (information
extraction), схожесть текстов (англ), различные классификации текстов (например,
по стилю или по эмоциональному содержанию). В рамках данной работы будет
рассмотрено улучшение решения этой задачи.
