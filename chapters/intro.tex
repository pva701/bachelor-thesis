\startprefacepage

Искусственные нейронные сети (ИНС) являются одним из мощных инструментов машинного
обучения.
Это математическая модель, которая работает подобно тому, как работает головной
мозг человека \cite{rosenblatt58a}.
Они были изобретены в 60-х годах предыдущего века\cite{rosenblatt58a}, и нашли свое активное
применение в наши дни. Конечно ИНС претерпевали изменения и модернизации, было разработано множество архитектур[статьи], а также алгоритмов обучения\cite{Duchi2011, zeiler2012, rprop93}.
Нейронные сети успешно решают задачи анализа и обработки изображений, текста и
звука[статьи].

%Актуальность
В последнее время ИНС используются для множества задач, связанных с Natural
Language Processing (NLP). Важной задачей NLP является задача 
обработки и извлечения семантического содержимого предложений (sentence modelling).
Она используется для создания искусственного интеллекта, чат-ботов,
переводов текстов (machine translation), извлечение фактов из текста (information
extraction), схожесть утверждений (semantic relatedness), 
а также различных классификаций текстов, например,
по стилю или по эмоциональному содержанию (sentiment analysis).
В частности, широкое распространение получили методы, которые сопоставляют предложению некоторый вещественный вектор.

%Новизна
В рамках данной работы будут предложены алгоритмы, которые используют \emph{дерево синтаксического разбора} (syntactic parse tree) предложения, а также учитывают локальный контекст каждого слова предложения[статьи]. На данный момент не существует решений, которые явно учитывают оба этих аспекта.

% Структура работы
В главе 1 будут введены вспомогательные понятия, сформулированы решаемые задачи.
Будут рассмотрены существующие решения на основе \emph{cверточных нейронных сетей} (от англ. Сonvolutional Neural Networks или CNN), \emph{рекурсивных нейронных сетей} (от англ. Recursive Neural Network или RNN), а также такие решения как Paragraph Vector и \emph{рекурсивная тензорная нейронная сеть} (от англ. Recursive Neural Tensor Network или RNTN).

В главе 2 будут рассмотрены предложенные решения, их принцип работы и обоснование.

В главе 3 будут рассмотрены результаты, достигнутые предложенными решениями на различных задачах, а также сравнения с существующими решениями.