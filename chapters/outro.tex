Были предложены две идеи под названиями: \textquote{локальный контекст слов} и \textquote{значимые поддеревья}.
Первая идея может использоваться как вспомогательный элемент в моделях основанных на синтаксических деревьях разбора, вторая идея может использоваться как самостоятельная модель подсчета векторного представления, а также как вспомогательный элемент для других моделей.

Обе идеи показали свою эффективность в связке с уже существующими решениями, причем, с предложенными улучшениями точность и F1-мера превосходят основопологающие решения, что доказывает эффективность этих идей.
На основе предложенных идей были построены архитектуры, которые превосходят текущие результаты в некоторых задачах. Время обучения предложенных архитектур сравнимо с существующими решениями.

В данной работе исследованы не все аспекты предложенных идей. Так в идее \textquote{локальный контекст слов} интерес представляет выбор слов для подсчета контекста, например, в пределах некоторого поддерева, а не всегда k-грамму, либо же использование в качестве контекста не только смежные слова.
В идее \textquote{значимые поддеревья} не была исследована возможность выбора в качестве $K$ некоторой функции, вид этой функции также остается открытым вопросом.