\documentclass[specification,annotation]{itmo-student-thesis}

\usepackage[utf8]{inputenc}
\usepackage{amsmath,amssymb}
\usepackage[russian]{babel}
\usepackage{multirow}

\usepackage{graphicx}
\graphicspath{ {images/} }
\selectlanguage{russian}

\newcommand{\todo}[1]{\vspace{5 mm}\par \noindent
\marginpar{ToDo}
\framebox{\begin{minipage}[c]{0.95 \textwidth}
    \tt #1 \end{minipage}}\vspace{5 mm}\par}

\addbibresource{bachelor-thesis.bib}

\begin{document}

\studygroup{M3439}
\title{Построение векторного представления предложения с использованием дерева синтаксического разбора}
\author{Пересадин Илья Валерьевич}{Пересадин И.В.}
\supervisor{Фильченков Андрей Александрович}{Фильченков А.А.}{доц. каф КТ, к.ф-м.н.}{}
\publishyear{2017}

\startdate{01}{сентября}{2016}
\finishdate{31}{мая}{2017}
\thedefencedate{}

\technicalspec{
    В рамках данной работы требуется разработать модель для построения векторного представления предложения
    на основе дерева синтаксического разбора. 
    Требуется изучить существующие методы и сравнить их предложенным решением.
}

\plannedcontents{
  \begin{enumerate}
    \item Постановка задачи и обзор предметной области
    \item Описание предложенного решения
    \item Экспериментальная проверка метода и сравнение с существующими моделями
  \end{enumerate}
}

\plannedsources{
 \begin{enumerate}
    \item Yoon Kim, Convolutional Neural Networks for Sentence Classification
    \item Richard Socher et al, Recursive Deep Models for Semantic Compositionality
Over a Sentiment Treebank
  \end{enumerate}
}

\addstage{Ознакомление с предментой областью}{01.02.2017}
\addstage{Разработка моделей, основанных на архитектурах LSTM и CNN}{01.03.2017}
\addstage{Реализация и тестирование придуманных моделей}{10.03.2017}
\addstage{Разработка моделей, основанных на дереве синтаксического разбора}{01.04.2017}
\addstage{Тестирование моделей на основе дерева синтаксического разбора}{15.04.2017}
\addstage{Написание пояснительной записки}{30.05.2017}

\researchaim{
Разработать модель для построения векторного представления предложения на основе дерева синтаксического разбора}
  
\researchtargets{
  \begin{enumerate}
    \item формулирование проблем, возникающих при построении векторного представления предложения существующими методами
    \item реализация методов, которые решают эти проблемы
  \end{enumerate}
}
  
\advancedtechnologyusage{
    Были использованы язык программирования Python, библиотеки numpy, scikit-learn и matplotlib,
    интегрированная среда разработки PyCharm, фрэймворк для символьных вычислений Tensorflow,
    система контроля версий Git и система компьютерной верстки LaTeX.
}

\researchsummary{
    Разработана модель для построения векторного представления предложения на основе дерева синтаксического разбора
}

\researchfunding{
    Грантов или других форм государственной поддержи и субсидирования в процессе работы не предусматривалось.
}
 
\researchpublications{
    Работа не была опубликована.
}

\secretary{}

%% Эта команда генерирует титульный лист и аннотацию.
\maketitle{Бакалавр}

% Оглавление
\tableofcontents

% Chapters
\startprefacepage
В нашу эпоху современных технологий, человек пытается максимально
роботизировать все процессы нашей жизни. Ученые работают над тем,
чтобы компьютер без проблем понимал, что от него хочет пользователь и с легкостью
решал поставленные задачи. 
Для того, чтобы компьютер мог понимать человеческий язык, было изобртено направление искусственного интеллекта~--- обработка естественного языка (от англ. Natural Language Processing). 
NLP изучает проблемы анализа и синтеза естественных языков\cite{wikinlp}.
Большую нишу NLP заняли искусственные нейронные сети.

Искусственные нейронные сети (ИНС) являются одним из мощных инструментов машинного
обучения.
Это математическая модель, которая работает подобно тому, как работает головной
мозг человека \cite{rosenblatt58a}.
Они были изобретены в 60-х годах предыдущего века, и нашли свое активное
применение в наши дни. Конечно ИНС претерпевали изменения и модернизации, было разработано множество архитектур[статьи], а также алгоритмов обучения\cite{Duchi2011, zeiler2012, rprop93}.

%Актуальность
Важной задачей NLP является задача  обработки и 
извлечения семантического содержимого предложения (sentence modelling).
Она используется для создания чат-ботов, переводов текстов (machine translation), 
извлечение фактов из текста (information extraction), схожесть утверждений (semantic relatedness), 
а также различных классификаций текстов, например, по стилю или по эмоциональному тону (sentiment analysis).
В частности, широкое распространение получили методы, 
которые сопоставляют предложению некоторый вещественный вектор, 
с помощью которого и происходит анализ предложения.

%Новизна
В данной работе будет предложена модель, вычисляющая векторное представление предложения, 
которая использует \emph{дерево синтаксического разбора} (syntactic parse tree) предложения, а также учитывают локальный контекст каждого слова предложения.
На данный момент не существует решений, которые явно учитывают оба этих аспекта.
Также будет предложен новый подход вычисления векторного представления предложения.

% Структура работы
В главе 1 будут введены вспомогательные понятия, сформулированы решаемые задачи.
Будут рассмотрены существующие решения на основе \emph{cверточных нейронных сетей} (от англ. Сonvolutional Neural Networks или CNN), \emph{рекурсивных нейронных сетей} (от англ. Recursive Neural Network или RNN), а также такие решения как Paragraph Vector и \emph{рекурсивная тензорная нейронная сеть} (от англ. Recursive Neural Tensor Network или RNTN).

В главе 2 будут рассмотрены предложенные улучшения, их принцип работы и обоснование.

В главе 3 будут рассмотрены результаты, достигнутые предложенными решениями на задачах классификации
эмоционального тона предложения и задача классификации вопросов, а также сравнение с существующими решениями.

%-*-coding: utf-8-*-

\chapter{Обзор предметной области}

\section{Вспомогательные понятия}

\subsection{Vector representation и Word embedding}
Vector representation (Векторное представление)~--- подход в машинном обучении, при котором некоторой сущности
сопоставляется вектор вещественных чисел.

Формально, $X$~--- множество объектов, тогда фунция $v(x):X \rightarrow R^n$
задает vector representation для объектов из множества $X$.

Причем похожим сущностям сопоставляются близкие
по некоторой метрике вектора, а различным ~--- удаленные друг от друга.

Word embedding ~--- это подход в Natural Language Processing (NLP), который
состоит в отображении слов некоторого словаря в $R^n$ с сохранением
семантических отношений между словами. 
То есть, например, некоторые компоненты вектора могут отвечать за пол объекта,
его одушевленность, съедобность и т.д. Значения компонент не предзадаются, а самоопределяются в результате обучения.


Word embedding обычно предобучают на достаточно большом корпусе текстовых
данных, а затем используют в задачах NLP. Существуют наиболее крупные корпусы слов с соответствующими word embedding, такие как word2vec и glove, которые также будут использоваться в данной работе.

\todo{Визуализация Word Embedding}

\section{Классификация}

Классификация ~-- один из разделов машинного обучения, посвященный решению
задачи классификации.

\subsection{Задача классификации}

Задача классификации~--- имеется множество объектов, каждый из которых принадлежит
к какому-то классу, количество классов чаще всего ограничено.
Существует обучающая выборка~--- множество объектов, метки
класса которых нам известны. Классовая принадлежность остальных объектов
неизвестна. Задача заключается в построении алгоритма, способного
классифицировать (присвоить метку класса) произвольный объект из исходного множества.

Формально, $X$~--- множество объектов, $Y$~--- множество классов,
существует отношение $y* : X \rightarrow Y$, заданное только для обучающей выборки.
Необходимо построить такой алгоритм $a: X \rightarrow Y$, способный для произвольного
$x \in X$ найти $y \in Y$.	

\section{Численная оценка качества классификации}

\subsection{Accuracy}
Accuracy~--- точность классификации. Самый простой способ оценки эффективности классификатора.
$$Accuracy =\frac{P}{N}$$
$P$~--- количество верно классифицированных объектов\\
$N$~--- количество объектов в выборке

\section{Существующие решения}

В данной главе будут приведены существующие решения задачи sentence modelling, для сравнения с предложенным решением.

\todo{Тут какие-то более простые вещи, Paragraph Vector}

\subsection{RNN и LSTM}
Основным инструментом для обработки текстов являются так называемые рекурсивные нейронные сети (Recursive Neural Network).

Рекурсивные нейронные сети предназначены для обработки последовательных данных, таких как звук, текст. В традиционных нейронных сетях все входы считаются независимыми друг от друга, но для многих задач это не является правдой.

Рекурсивные нейронные сети принимают слова последовательности поочереди, сохраняя внутри себя контекст уже принятого текста. Рекурсивными они называются потому что выполняют одну и ту же задачу для каждого элемента последовательности (а конкретно слов в тексте). Они достаточно хорошо отражают процесс восприятия информации человеком: после того как мы прочли начало предложение, в нашей голове уже сформировался некоторый контекст, и следующее слово обрабатывается нами с учетом уже прочитанной информации, а не воспринимается с чистого листа.

\begin{figure}[h]
\includegraphics[scale=0.7]{rnn}
\caption{\textbf{RNN}}
\label{fig:figure1}
\end{figure}

\begin{figure}[h]
\includegraphics[scale=0.7]{rnn-unfold}
\caption{\textbf{RNN в развернутом виде}}
\label{fig:figure2}
\end{figure}

\noindent $U, W, V$~--- параметры RNN сети\\
$x_t$~--- вектор, соответствующий слову $t$ \\
$s_t$~--- информация о первых $t$ словах \\
$o_t$~--- выходной вектор


%-*-coding: utf-8-*-

\chapter{Описание предложенного решения}

В данной главе будут рассмотрены предложенные идеи архитектур нейронных сетей на основе дерева разбора предложения.

В разделе \ref{ideas} будут кратко описаны предложенные идеи и мотивация их использования.

В разделе \ref{formal} будет дано формальное описание модели.

\section{Предложенные идеи} \label{ideas}

\subsection{Контекст слов в предложении}

Минусом подхода РТНС является то обстоятельство, что для поддеревьев с небольшим количеством слов (до 4-5 слов), методу сложно построить векторное представление достаточно точно, ввиду отсутствия контекста использования слов.\\
Так, в данном примере классификации эмоционального тона:

\begin{figure}[h]
\includegraphics[scale=0.6]{Context_Example2}
\caption{\textbf{Пример зависимости от контекста}}
\label{fig:context_ex}
\end{figure}
Слово {\fontfamily{cmss}\selectfont \textbf{constructed}} не несет смысловой нагрузки для задачи, и поэтому для него сложно построить точное векторное представление относительно решаемой задачи.

Однако в контексте слов {\fontfamily{cmss}\selectfont \textbf{than Memento}} становится понятно, что это сравнительная фраза, и построить векторное представление в контексте задачи становится проще.

Данная проблема наталкивает на мысль: учитывать слова, 
использующиеся вместе с каждым словом в предложении, перед передачей в РНТС.

\subsection{Значимые слова в предложении}
Еще одной проблемой существующих решений, является то,  что они учитывают все слова предложения, 
хотя достаточно большое количество слов не несет смысловой нагрузки для решаемой задачи, а также для построения взаимосвязей
между словами.
Это требует от модели более избирательного анализа слов, а также порождает проблему \textquote{затухания градиента}, из-за 
длинной последовательности вычислений.

Рассмотрим следующий примеры для задачи классификации эмоционального тона предложения на датасете, состоящем из
обзоров кинофильмов.

{\fontfamily{cmss}\selectfont A \textbf{screenplay more ingeniosly} constructed \textbf{than Memento}.}

{\fontfamily{cmss}\selectfont Suffers from the \textbf{lack} of a \textbf{compelling} or \textbf{comprehensible narrative}.}

{\fontfamily{cmss}\selectfont Still, this \textbf{flick is fun} and \textbf{host to} some truly \textbf{excellent sequences}.}

\noindent Жирным выделены слова, которе несут основную смысловую нагрузку предложения.\\
Можно видеть, что достаточно большой процент от общего количества слов в предложении составляют 
\textquote{незначимые} для задачи слова.

Данная идея как раз и состоит в выделении наиболее значимых слов и фраз в предложении.
Эту идею можно обобщить на дерево разбора.

\section{Формальное описание алгоритма} \label{formal}

\subsection{Предобработка данных} \label{prework}
Для того чтобы алгоритм смог проанализировать предложение, 
его необходимо привести к определенному формату. Мы рассмотрим обработку
тренировочных данных, обработка тестовых данных производится аналогично.

В первую очередь, предложение необходимо разбить на слова. 
Это можно сделать используя регулярные выражения. 

Затем по данному корпусу предложений необходимо построить словарь слов.
Словарь~--- это некоторое подмножество слов предложений корпуса.
Это делается для того, чтобы ограничить объем вычислений, так как использование всех
слов языка достаточно ресурсоемко. Обычно в словарь попадают слова, 
имеющие наибольшую частоту использования в данном корпусе, 
либо просто все слова корпуса.

Те слова из корпуса, которые не попали в словарь заменяются специальным словом $UNK$, 
оно также добавляется в словарь.

Теперь необходимо для предложения построить дерево синтаксического разбора.
В рамках данной работы мы будем делать это, используя готовое решение \cite{lex-parser}.

При тестировании модели, производится аналогичная предобработка, 
с тем лишь отличием, что используется словарь, построенный по тренировочным данным.

Заметим, что весь процесс предобработки предложений автоматизирован, 
и не требует вмешательства человека.

\subsection{Построение векторного представления слов} \label{word_embedding}
После того, как построен словарь, каждому слову $w$ из словаря сопоставляется целочисленный индекс $i_w$ 
от $0$ до $N-1$, где $N$~--- размер словаря.
Каждое слово в словаре задается вектором размерности $N$, 
в котором на позиции $i_w$ находится единица, а на остальных позиция~--- нули.

Оперированием данным вектором напрямую в вычислениях~--- ресурсоемко, поэтому вводится вещественная матрица $W$, размерности $N \times d$, где $N >> d$.\\
Тогда вектор для слова $w$ вычисляется как
$$a_w = E_{i_w} \cdot W$$
где $E$~--- единичная матрица $N \times N$.\\
Таким образом мы построили векторное представление для слов из словаря, которое задается матрицей $W$.

Матрица $W$ является обучаемым параметром, и инициализируется либо произвольными числами, 
либо же используются предобученные на больших корпусах данных векторные представления, 
такие как word2vec или GLoVe\cite{wor2vec, glove}.

\subsection{Описание идеи \textquote{локальный контекст слов}} \label{loc_context}

Формально, предложение задается матрицей в  $\mathbb{R}^{l \times{} d}$, в которой $i$-я строка равна векторному представлению слова на позиции $i$,\\
где $l$~--- количество слов в предложении,\\
$d$~--- размерность векторного представления слов. 

Наша задача: получить новую матрицу в $\mathbb{R}^{l \times {} t}$, где $i$-я строка задает векторное представление \textit{контекста} данного слова в предложении. 
Данная операция задается оператором $F:\mathbb{R}^{l \times d} \to \mathbb{R}^{l \times t}$.

В рамках данной работы мы рассмотрим такие операторы $F$, которые в качестве контекста
для слова $i$ используют $k$-грамму с началом в $i$.\\
$k$-грамма~--- это слова на таких позициях $j$, что $i \le j < i + k$, где $k \in \mathbb{N}$~--- фиксированное значение и является параметром алгоритма.\\
Оператор $F$ такого вида определяется оператором
$C:\mathbb{R}^{k \times d} \to \mathbb{R}^t$, который определяет способ 
получить векторное представление из смежных слов. \\
Тогда $$F(X)_i = C(X'_{i..i+k-1})$$
где $X'$~--- это $X$ c добавленными в конец $k-1$ строками из нулей, \\
$X'_{i..i+k-1}$~--- матрица, образованная строками $X'$ c $i$ по $i+k-1$.\par

\vspace{5mm}

\noindent В данной работе будут использоваться следующие операторы $C$:
\begin{itemize}
    \item{полносвязный слой}
        $$C_{FC}(X)=\sigma(X^T \cdot W + b)$$
        где $W \in R^k, b \in R^d$, в данном случае $t=d$
    \item{сверточная нейронная сеть}
        $$c_j(X)=\sigma(X \odot m_j + b_j)$$
        $$C_{CNN}(X)=[c_1(X); c_2(X); \cdots c_t(X)]$$
        где $m_j \in \mathbb{R}^{k \times d}, b_j \in \mathbb{R}$,\\
        $\odot$~--- поэлементное произведение и суммирование полученных произведений, 
        так называемая операция \textquote{свертки}
    \item{долгая краткосрочная память}
    $$h_i=LSTM_h(c_{i-1},h_{i-1}, X_i) \text{ для } i \text{ от } 1 \text { до } k$$  
    $$c_i=LSTM_c(c_{i-1}, h_{i-1}, X_i) \text{ для } i \text{ от } 1 \text { до } k$$ 
    $$c_0 = \emptyset, h_0 = \emptyset$$
    $$C_{LSTM}(X) = h_k$$
    где $LSTM$~--- ячейка долгой краткосрочной памяти, описанной в разделе \ref{lstm} \\
    $LSTM_h, LSTM_c$~--- вычисление векторов $c$ и $h$ из $LSTM$ ячейки соответственно,\\
    $h_i \in \mathbb{R}^t, c_i \in \mathbb{R}^t$
\end{itemize}

\noindent Эта идея была названа \textquote{локальный контекст слов}.

\subsection{Описание идеи \textquote{значимые поддеревья}} \label{mean_subtree}

Перед нами стоит задача: вычислить для каждой вершины дерева разбора 
векторное представление фразы, соотвествующей этой вершине.
Мы будем делать это восходящим образом, вычисляя векторное представление для листьев,
затем для их предков, и так далее до корня дерева.

Обозначим векторное представление вершины $v$ за $f_v \in \mathbb{R}^s$.
Для вычисления $f_v$ в поддереве вершины $v$ будем выбирать такие поддеревья, 
что соответствующие им фразы являются наиболее значимыми для решаемой задачи.

Введем для каждой вершины $u$ весовой вектор $w_u \in \mathbb{R}^p$  так, что
чем больше квадрат нормы  этого вектора, тем более значима фраза, соотвествующая $u$.

Теперь чтобы посчитать векторное представление вершины $v$, выберем $K(v)$ вершин
в ее поддереве с наибольшими значениями $\lVert w_u \rVert^2$, пусть это $\{ u'_1, u'_2, \dots u'_{K(v)} \}$.
Затем с помощью некоторого оператора $G_v:\mathbb{R}^{K(v) \times s} \to \mathbb{R}^s$, 
передав в него выбранные значения $f_{u'_i}$, посчитаем векторное представление $f_v$.
Теперь остается пересчитать $w_v$. Сделаем это аналогичным образом, используя некоторый оператор 
$W_v :\mathbb{R}^{K(v) \times (s + p)} \to \mathbb{R}^p$, $f_{u'_i}$ и $w_{u'_i}$.

Формально:
$$TopK_v \{ \lVert w_{u_1} \rVert^2, \lVert w_{u_2} \rVert^2, \dots, \lVert w_{u_{2n-1}} \rVert^2\} = \{u'_1, u'_2, \dots, u'_{K(v)}\}$$
$$f_v = G_v(f_{u'_1}, f_{u'_2}, \dots, f_{u'_{K(v)}})$$
$$w_v = W_v(f_{u'_1} \circ w_{u'_1},f_{u'_2} \circ w_{u'_2}, \dots, f_{u'_{K(v)}} \circ w_{u'_{K(v)}})$$

\noindent $n$~--- количество слов в фразе, соответствующей вершине $v$\\
$u_1, u_2, \dots u_{2n-1}$~--- вершины поддерева $v$ в порядке эйлерова обхода\\
$TopK_v$~--- функция, которая выбирает $K(v)$ наибольших значений и возвращает их порядковые номера\\
$K(v)$~--- функция, которая определяет количество значимых поддеревьев для вершины $v$\\
$\circ$~--- операция конкатенации двух векторов в один

Мы можем видеть, что данный подход задается семействами операторов $G_v$ и $W_v$, и функцией $K$.

Также заметим, что если у нас уже есть вектора $f_v$, посчитанные некоторой другой моделью, не зависящей от данного подхода,
мы можем с помощью данного метода посчитать еще одни вектора $f'_v$ используя $f_v$, просто заменив во второй формуле $f_v$ на $f'_v$. И для предсказания использовать пару $(f_v, f'_v)$. Здесь нам по сути важен только $f'_{root}$, так как $f'_v$ не участвует в рекурсивных вычислениях.

Эта идея была названа \textquote{значимые поддеревья}.

\subsection{Архитектура и обучение модели} \label{arch}

Архитектуру полученной модели можно условно разбить на три части
\begin{enumerate}
    \item{подсчет векторного представления локального контекста}
    \item{подсчет векторного представления поддеревьев}
    \item{способ решения поставленной задачи по векторному представлению корня}
\end{enumerate}

Первый пункт был формально описан в разделе \ref{loc_context}

Раздел \ref{mean_subtree} описывает предложенный механизм для подсчета векторного представления поддеревьев.
Помимо предложенного решения, в рамках данной работы используется простая рекуррентная модель пересчета поддеревьев \cite{SocherEtAl2011:RNN}, которая задается как:
$$f_v = \sigma(f_l \cdot W_1 + f_r \cdot W_2 + b)$$
где $f_v$~--- векторное представление вершины $v$, а $f_l$ и $f_r$~--- непосредственных
потомков веришы $v$,\\
$W_1, W_2 \in \mathbb{R}^{s \times s}$,
$b \in \mathbb{R}^s$~--- параметры рекуррентной модели

А также, древовидная LSTM модель \cite{DBLP:journals/corr/TaiSM15}, которая задается как:
$$\tilde{i}_v=\sigma \left( U_1^{(i)} \cdot h_{v,1} + U_2^{(i)} \cdot h_{v,2} + b^{(i)} \right)$$
$$\tilde{f}_{vk}=\sigma \left( U_1^{(f)} \cdot h_{v,1} + U_2^{(f)} \cdot h_{v,2} + b^{(f)} \right),\text{ }k=1,2$$
$$\tilde{o}_{v}=\sigma \left( U_1^{(o)} \cdot h_{v,1} + U_2^{(o)} \cdot h_{v,2} + b^{(o)} \right)$$
$$\tilde{u}_{v}=\tanh \left( U_1^{(u)} \cdot h_{v,1} + U_2^{(u)} \cdot h_{v,2} + b^{(u)} \right)$$
$$\tilde{c}_v=\tilde{i}_v \odot \tilde{u}_v + \tilde{f}_{v,1} \cdot \tilde{c}_{v, 1} + \tilde{f}_{v,2} \cdot \tilde{c}_{v, 2}$$
$$\tilde{h}_v=\tilde{o}_v \cdot \tanh(\tilde{c}_v)$$

Способ решения задачи по векторному представлению корня непосредственно зависит от самой задачи.
То есть для каждой, отдельно взятой задачи, строится модель $G$, которая задается набором параметров $S$.
Модель $G$ принимает на вход $f_{root}$ и возвращает значение из области значений решаемой задачи. 
Набор параметров $S$ тренируется вместе с параметрами предложенной модели. Построение $G$ выходит за рамки данной работы, так как зависит от задачи, но мы рассмотрим $G$ для задач, использующихся в данной работе.

В данной работе нам потребуется решать задачу классификации предложения.
Для нее будет использоваться полносвязный слой c последующей softmax-функцией активации. 
В качестве функции потери используется кросс-энтропийная функции ошибки.\\
Формально:
$$z=f_{root} \cdot W_{out} + b_{out}$$
$$\tilde{y}_j=\frac{e^{z_j}} {\sum_{k=1}^K e^{z_k}}, j=1..K$$
$$L_2 = -\sum_{i=1}^K y_i \cdot \log(\tilde{y_i})$$
где $f_{root} \in R^s$~--- векторное представление корня\\
$K$~--- количество классов в задаче классификации\\
$W_{out} \in \mathbb{R}^{s \times K}, b_{out} \in \mathbb{R}^K$~--- параметры полносвязного слоя\\
$y_j$~--- правильный ответ для данного входного примера\\
$\tilde{y}_j$~--- распределение вероятностей по классам задачи классификации, сгенерированное моделью\\
$L_2$~--- значение ошибки на данном входном примере.\\
Оптимизация модели состоит в минимизации суммы $L_2$ по всем тренировочным данным.

\subsection{Архитектурные решения}
В данной работе были протестированы различные архитектуры:

\textbf{C подсчетом локального контекста слов}\\
В данной архитектуре варировались операторы, вычисляющие локальный контекст окна, размер окна $k$ и функция, вычисляющая векторные представления поддеревьев.

\textbf{C пересчетом поддеревьев методом значимых поддеревьев}\\
В данной архитектуре варировались операторы, вычисляющие векторное представление поддерева по значимым поддеревьям.

\textbf{С обоими предложенными подходами}\\
В данной архитектуре варировались операторы, вычисляющие локальный контекст окна, размер окна $k$.

\vspace{5mm}
Также была протестирована архитектура, которая использует векторное представление предложенного алгоритма как вспомогательный вектор, передающийся в $LSTM$ для предложения, а выходной вектор $LSTM$ используется, как описано в конце раздела \ref{arch}.
%-*-coding: utf-8-*-

\chapter{Эксперименты и результаты}
 
\subsection{Описание наборов данных и методики тестирования}
 
Предложенное решение было протестировано на нескольких задачах. 
Наборы данных содержат синтаксические деревья либо предложения. 
Для предложений были построены синтаскические деревья с помощью[ссылка].
\vspace{5mm}

\noindent \begin{minipage}{\linewidth}
\captionof{table}{\textbf{Наборы данных}} \label{tab:title} 
\begin{tabular}{|c|c|c|c|c|}
\hline
\multirow{2}{*}
  {Название}      & Кол-во         & Средняя длина          & Кол-во          & Кол-во  \\
                  & классов        & предложения            & трен. примеров  & тест. примеров \\ \hline
\textbf{MR}       & 2              & 20                     &  10662          &  CV      \\ \hline
\textbf{SST-1}    & 5              & 18                     &  11855          &  2210    \\ \hline
\textbf{SST-2}    & 2              & 19                     &  9613           &  1821    \\ \hline
\textbf{Subj}     & 2              & 23                     &  10000          &  CV     \\ \hline
\textbf{TREC}     & 6              & 10                     &  5952           &  500    \\ \hline
\end{tabular}
\vspace{5mm}

\begin{itemize}
\item{\textbf{MR}} ~--- набор данных с обзорами фильмов, содержит предложения\\
\item{\textbf{SST-1}, \textbf{SST-2}}~--- наборы данных с обзорами фильмов, содержат синтаксические деревья разбора\\
\item{\textbf{Subj}}~--- набор данных с субъективными и объективными утверждениями, содержит предложения\\
\item{\textbf{TREC}}~--- набор данных с шестью типами вопросов, содержит предложения
\end{itemize}
\end{minipage}
\vspace{5mm}

Для обучения был использован язык Python 3.4 и фреймворк символьных вычислений Tensorflow[ссылка].
Обучение проводилось с помощью оптимизаторов Adam[ссылка] и Adagrad[ссылка].

В процессе обучения архитектур была использована $L_2$ регуляризация[ссылка].
Это подход, когда к минимизируемой ошибке добавляются квадраты всех параметров сети, умноженных на некоторый коэффициент.
Он предотвращает переобучение параметров, так как параметры не могут принимать большие значения из-за увеличения ошибки.
$L_2$ регуляризация задается коэффициентом регуляризации $\lambda$.

Кроме того, использовался dropout (дропаут)[ссылка]. 
Эта техника заключается в том, что во время обучения некоторые нейроны с некоторой вероятностью не участвуют в предсказании.
Этот подход предотвращает cоадоптацию нейронов. Он задается вероятностью участия нейрона в предсказании $p$.

Обучение происходило минибатчами. Один минибатч включает в себя несколько тренировочных примеров, ошибка и градиент оптимизатора
берется как среднее ошибок и среднее градиентов по примерам в минибатче. Это сделано для более плавного роста ROC прямой и, следовательно, более быстрой сходимости. Размер минибатча составлял 25.

В датасетах \textbf{SST-1} и \textbf{SST-2} проаннатированы все поддеревья, 
поэтому для увеличения эффективности обучения ошибка учитывается не только от корня дерева, но и от всех поддеревьев.

\subsection{Тестирование архитектуры с локальными контекстами}
В данном разделе мы рассмотрим эксперименты над архитектурами, с вычислением локальных контекстов.

Сначала рассмотрим использование простой рекурсивной модели в качестве механизма пересчета поддеревьев.

\vspace{5mm}
\noindent \begin{minipage}{\linewidth}
\captionof{table}{\textbf{Вычисление локальных контекстов на основе CNN}} \label{tab:title} 
\begin{tabular}{|c|c|c|c|c|c|}
\hline
\multirow{2}{*}{Набор}   &                \multicolumn{5}{c|}{Размер k-граммы} \\ \cline{2-6} 
     данных              &   \textbf{2} & \textbf{3} & \textbf{4} & \textbf{5} & \textbf{2,3,4,5} \\ \hline
\textbf{MR}              &              &            &            &            &  \\ \hline
\textbf{SST-1}           & 46.7\%       & 46.3\%     &  46.1\%    &  46.1\%    &  47.2  \\ \hline
\textbf{SST-2}           & 2            & 19         &  9613      &  1821      & \\ \hline
\textbf{Subj}            & 2            & 23         &  10000     &  CV        & \\ \hline
\textbf{TREC}            & 6            & 10         &  5952      &  500       & \\ \hline
\end{tabular}
\end{minipage}
\vspace{5mm}

\noindent \begin{minipage}{\linewidth}
\captionof{table}{\textbf{Вычисление локальных контекстов на основе LSTM}} \label{tab:title} 
\begin{tabular}{|c|c|c|c|c|c|}
\hline
\multirow{2}{*}{Набор}   &                \multicolumn{5}{c|}{Размер k-граммы} \\ \cline{2-6} 
     данных              &   \textbf{2} & \textbf{3} & \textbf{4} & \textbf{5} & \textbf{2,3,4,5} \\ \hline
\textbf{MR}              & 2            & 20         &  -     &  CV        &                      \\ \hline
\textbf{SST-1}           & 46.4\%       & 46.4\%     &  46.3\%    &  46.2\%    & \\ \hline
\textbf{SST-2}           & 2            & 19         &  9613      &  1821      & \\ \hline
\textbf{Subj}            & 2            & 23         &  10000     &  CV        & \\ \hline
\textbf{TREC}            & 6            & 10         &  5952      &  500       & \\ \hline
\end{tabular}\\
\vspace{5mm}
\end{minipage}
\vspace{5mm}

\todo{График тут мб?}

Последний столбец описывает модель, которая считает векторное представление $k$-грамм для $k=2,3,4,5$ и конкатенирует их.

Подход превосходит базовую модель РНТС на датасетах \textbf{SST-1} и \textbf{SST-2}, 
которая использует простой рекурсивный подсчет поддеревьев.

Мы можем видеть, что вычисление на основе CNN лучше работает на небольших размерах $k$-грамм.
Вычисление на основе LSTM немного лучше приспосабливается к большим окнам, хотя как и CNN ухудшает результат.
Это происходит потому что в контекст слов начинают попадать далекие слова, 
которые могут иметь достаточно произвольный характер.

Также можем видеть, что учет $2,3,4,5$-грамм значительно улучшает результат.

Несмотря на достаточно схожие результаты на основе CNN и LSTM, 
подход с CNN обучается значительно быстрее, нежели подход с LSTM, так как требует меньше вычислений.
Несмотря на это, LSTM-подход более устойчив к переобучению, и не требует кропотливой подборки $\lambda$ и $p$.

\todo{тут еще Tree-LSTM}

\subsection{Тестирование архитектуры со значимыми поддеревьями}

Напомним, что в подходе со значимыми поддеревьями, количество значимых поддеревьев определяется функцией $K$.

Сначала будут рассмотрены модели с постоянным значением $K$, не зависящим от размера поддерева.

\vspace{5mm}
\noindent \begin{minipage}{\linewidth}
\captionof{table}{\textbf{Постоянное количество значимых поддеревьев}} \label{tab:title} 
\begin{tabular}{|c|c|c|c|c|c|}
\hline
\multirow{2}{*}{Набор}   &                \multicolumn{5}{c|}{Количество значимых поддеревьев} \\ \cline{2-6} 
     данных              &   \textbf{2} & \textbf{3} & \textbf{4} & \textbf{6} & \textbf{8} \\ \hline
\textbf{MR}              &              &            &            &            &  \\ \hline
\textbf{SST-1}           & 46.7\%       & 46.3\%     &  46.1\%    &  46.1\%    &  47.2  \\ \hline
\textbf{SST-2}           & 2            & 19         &  9613      &  1821      & \\ \hline
\textbf{Subj}            & 2            & 23         &  10000     &  CV        & \\ \hline
\textbf{TREC}            & 6            & 10         &  5952      &  500       & \\ \hline
\end{tabular}
\end{minipage}
\vspace{5mm}


\startconclusionpage

Были предложены две идеи под названиями: \textquote{локальный контекст слов} и \textquote{значимые поддеревья}.
Первая идея может использоваться как вспомогательный элемент в моделях основанных на синтаксических деревьях разбора, вторая идея может использоваться как самостоятельная модель подсчета векторного представления, а также как вспомогательный элемент для других моделей.

Обе идеи показали свою эффективность в связке с уже существующими решениями, причем, с предложенными улучшениями точность и F1-мера превосходят основопологающие решения, что доказывает эффективность этих идей.
На основе предложенных идей были построены архитектуры, которые превосходят текущие результаты в некоторых задачах. Время обучения предложенных архитектур сравнимо с существующими решениями.

В данной работе исследованы не все аспекты предложенных идей. Так в идее \textquote{локальный контекст слов} интерес представляет выбор слов для подсчета контекста, например, в пределах некоторого поддерева, а не всегда k-грамму, либо же использование в качестве контекста не только смежные слова.
В идее \textquote{значимые поддеревья} не была исследована возможность выбора в качестве $K$ некоторой функции, вид этой функции также остается открытым вопросом.

\printmainbibliography

\end{document}
