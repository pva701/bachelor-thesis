
\subsection{Мешок слов}
\todo{мб здесь написать подробнее, конкретное решение}

Одним из первых подходов для анализа текстовых данных, 
является так называемый мешок слов (от англ. bag of words)[ссылка].
Данный подход не учитывает ни грамматические закономерности между словами, ни даже порядок слов.
Предложение рассматривается как мультисет слов.

Метод получил некоторые усовершенствования, такие как $NBoW$ (Neural Bag of Words), 
в котором рассматриваются не только предложение, но фразы в предложении из одного, двух и более слов\cite{DBLP:journals/corr/KalchbrennerGB14}.






\subsection{Численные оценки качества задачи классификации}

Задача классификации формулируется следующим образом:

Пусть $X$~--- множество объектов, $Y$~--- множество классов,
существует отношение $y* : X \rightarrow Y$, заданное только для обучающей выборки.
Необходимо построить такой алгоритм $a: X \rightarrow Y$, способный для произвольного
$x \in X$ найти $y \in Y$.	

\textbf{Точность}\\
Точность (от англ. accuracy)~--- это один из самых простых способов оценить
решение задачи классификации. Точность вычисляется следующим образом.

$$Accuracy =\frac{P}{N}$$
$P$~--- количество верно классифицированных объектов\\
$N$~--- количество объектов в выборке

\textbf{Кросс-энтропийная функция ошибки}\\